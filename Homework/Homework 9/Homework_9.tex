\documentclass[12pt]{article}
 
\usepackage[margin=1in]{geometry}
\usepackage{amsmath,amsthm,amssymb, mathtools}
\usepackage[T1]{fontenc}
\usepackage{lmodern}
\usepackage{fixltx2e}
\usepackage[shortlabels]{enumitem}
\usepackage{pgfplots}                                                           
\usepackage{pgf}
\usepackage{tikz}                                                               
\usepackage{float}
\usepackage{graphicx}
\graphicspath{ {./} }  
 
\newcommand{\N}{\mathbb{N}}
\newcommand{\R}{\mathbb{R}}
\newcommand{\Z}{\mathbb{Z}}
\newcommand{\Q}{\mathbb{Q}}
\newcommand{\rpm}{\sbox0{$1$}\sbox2{$\scriptstyle\pm$}
  \raise\dimexpr(\ht0-\ht2)/2\relax\box2 }
\newcommand{\norm}[1]{\left\lVert#1\right\rVert}

 
\newenvironment{theorem}[2][Theorem]{\begin{trivlist}
\item[\hskip \labelsep {\bfseries #1}\hskip \labelsep {\bfseries #2.}]}{\end{trivlist}}
\newenvironment{lemma}[2][Lemma]{\begin{trivlist}
\item[\hskip \labelsep {\bfseries #1}\hskip \labelsep {\bfseries #2.}]}{\end{trivlist}}
\newenvironment{exercise}[2][Exercise]{\begin{trivlist}
\item[\hskip \labelsep {\bfseries #1}\hskip \labelsep {\bfseries #2.}]}{\end{trivlist}}
\newenvironment{problem}[2][Problem]{\begin{trivlist}
\item[\hskip \labelsep {\bfseries #1}\hskip \labelsep {\bfseries #2.}]}{\end{trivlist}}
\newenvironment{question}[2][Question]{\begin{trivlist}
\item[\hskip \labelsep {\bfseries #1}\hskip \labelsep {\bfseries #2.}]}{\end{trivlist}}
\newenvironment{corollary}[2][Corollary]{\begin{trivlist}
\item[\hskip \labelsep {\bfseries #1}\hskip \labelsep {\bfseries #2.}]}{\end{trivlist}}
\newcommand{\textfrac}[2]{\dfrac{\text{#1}}{\text{#2}}}

\begin{document}

\title{Numerical Analysis: Homework \#9}

\author{Chris Hayduk}
\date{\today}

\maketitle

\begin{problem}{1}
\end{problem}

\begin{enumerate}
	\item[a)] 0 is an eigenvalue of $A$ if $A\vec{v} = 0\vec{v}$ for some $\vec{v} \neq \vec{0}$.\\
	
	We can simplify this as,
	\begin{align*}
		A\vec{v} = 0\vec{v} = \vec{0}
	\end{align*}
	
	Since we know $A$ is invertible, the equation $A\vec{x} = \vec{0}$ only has the trivial solution $\vec{x} = \vec{0}$.\\
	
	This implies that $\vec{v} = 0$. However, we assumed that $\vec{v} \neq 0$, a contradiction. Thus, 0 is not an eigenvalue of $A$ when $A$ is invertible.
	
	\item[b)] Suppose $\lambda$ is a non-zero eigenvalue of $A$. Then, $\exists \vec{v} \neq \vec{0}$ such that $A\vec{v} = \lambda\vec{v}$.\\
	
	Now let's multiply $A^{-1}$ on both sides,
	\begin{align*}
		A^{-1}(A\vec{v}) &= A^{-1}(\lambda\vec{v})\\
		\vec{v} &= A^{-1}\lambda\vec{v}\\
		A^{-1}\vec{v} &= \lambda^{-1}\vec{v}
	\end{align*}
	
	This satisfies the definition of an eigenvalue. Thus, $\lambda^{-1}$ is an eigenvalue of $A^{-1}$ if $\lambda$ is an eigenvalue of $A$.
\end{enumerate}
\\
\begin{problem}{2}
\end{problem}

After coding a power method program in Python with the following initial vector:\\

$\vec{z}^{(0)} = \begin{bmatrix}1.0\\0.30683515\\0.50975501\end{bmatrix}$\\

I obtained the following output:\\

$\lambda^{(m)}_1 = 5.031308512575385$\\

$\vec{z}^{(m)} = \begin{bmatrix}0.87931865\\1.0\\-0.07802744\end{bmatrix}$\\
\\
\begin{problem}{3}
\end{problem}

We have,

\begin{align*}
	&\int_{a}^{b} \sqrt{1 + (3x^2)^2} dx\\
	&\int_{a}^{b} \sqrt{1 + 9x^4} dx
\end{align*}

Thus,

\begin{align*}
	f(x) = \sqrt{1 + 9x^4}
\end{align*}

From (5.22), we have,

\begin{align*}
	S_6(f) &= \frac{h}{3}[f(x_0) + 4f(x_1) + 2f(x_2) + 4f(x_3) + 2f(x_4) + 4f(x_5) + f(x_6)]\\
	&= \frac{b-a}{18}[\sqrt{1+9a^4} + 4\sqrt{1+9(a+h)^4} + 2\sqrt{1+9(a+2h)^4} + 4\sqrt{1+9(a+3h)^4} +\\
	&2\sqrt{1+9(a+4h)^4} + 4\sqrt{1+9(a+5h)^4} + \sqrt{1+9(a+6h)^4}]
\end{align*}
\\
\begin{problem}{4}
\end{problem}

From the error formula (5.35), we have,

\begin{align*}
	E_n(f) &= -\frac{h^4(b-a)}{180}f^{(4)}(c_n)\\
	&= -\frac{\frac{1-0}{n}^4(1-0)}{180}[4e^{-c_n^2}(4c_n^4 - 12c_n^2 + 3)]\\
	&= -\frac{\frac{1}{n}^4}{180}[4e^{-c_n^2}(4c_n^4 - 12c_n^2 + 3)]
\end{align*}

with $c_n \in [0, 1]$.\\

This expression is maximized at $c_n = 0$. This yields,

\begin{align*}
	|E_n(f)| &\leq 12\frac{\frac{1}{n}^4}{180}\\
	&= \frac{\frac{1}{n}^4}{15}\\
	&= \frac{1}{15n^4}
\end{align*}

Thus we have,

\begin{align*}
	|E_n(f)| \leq 10^{-8} &\implies \frac{1}{15n^4} \leq 10^{-8}\\
	&\implies 15n^4 \geq 10^8\\
	&\implies n^4 \geq \frac{10^8}{15}\\
	&\implies n \geq 50.81327
\end{align*}

Thus, we must choose $n \geq 51$ to ensure that the absolute value of the error is less than $10^{-8}$.

After coding a Python function for Simpson's Rule, I approximated the integral using $n = 51$. This yielded an approximation of 

\begin{align*}
	&\int_{0}^{1} e^{-x^2} dx \approx S_{51}(f) = 0.7443725443100901
\end{align*}
\end{document}