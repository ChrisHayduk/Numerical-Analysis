\documentclass[12pt]{article}
 
\usepackage[margin=1in]{geometry}
\usepackage{amsmath,amsthm,amssymb, mathtools}
\usepackage[T1]{fontenc}
\usepackage{lmodern}
\usepackage{fixltx2e}
\usepackage[shortlabels]{enumitem}
\usepackage{pgfplots}
 
\newcommand{\N}{\mathbb{N}}
\newcommand{\R}{\mathbb{R}}
\newcommand{\Z}{\mathbb{Z}}
\newcommand{\Q}{\mathbb{Q}}
 
\newenvironment{theorem}[2][Theorem]{\begin{trivlist}
\item[\hskip \labelsep {\bfseries #1}\hskip \labelsep {\bfseries #2.}]}{\end{trivlist}}
\newenvironment{lemma}[2][Lemma]{\begin{trivlist}
\item[\hskip \labelsep {\bfseries #1}\hskip \labelsep {\bfseries #2.}]}{\end{trivlist}}
\newenvironment{exercise}[2][Exercise]{\begin{trivlist}
\item[\hskip \labelsep {\bfseries #1}\hskip \labelsep {\bfseries #2.}]}{\end{trivlist}}
\newenvironment{problem}[2][Problem]{\begin{trivlist}
\item[\hskip \labelsep {\bfseries #1}\hskip \labelsep {\bfseries #2.}]}{\end{trivlist}}
\newenvironment{question}[2][Question]{\begin{trivlist}
\item[\hskip \labelsep {\bfseries #1}\hskip \labelsep {\bfseries #2.}]}{\end{trivlist}}
\newenvironment{corollary}[2][Corollary]{\begin{trivlist}
\item[\hskip \labelsep {\bfseries #1}\hskip \labelsep {\bfseries #2.}]}{\end{trivlist}}
\newcommand{\textfrac}[2]{\dfrac{\text{#1}}{\text{#2}}}

\begin{document}

\title{Numerical Analysis: Homework \#4}

\author{Chris Hayduk}
\date{\today}

\maketitle

\begin{problem}{1}
\end{problem}

\noindent Suppose that there are two polynomials $P_n(x)$ and $Q_n(x)$ of degree $\leq n$ which interpolate our $n+1$ data points $(x_0, y_0), (x_1, y_1), ..., (x_n, y_n)$. Define:
\begin{center}
$R(x) = P_n(x) - Q_n(x)$
\end{center}
Since $P_n$ and $Q_n$ both have degree $\leq n$, we also have deg($R$) $\leq n$. Furthermore, observe that, by the properties of interpolation, we have:
\begin{center}
$R(x_i) = P_n(x_i) - Q_n(x_i) = y_i - y_i = 0$
\end{center}
for $i = 0, ..., n$. Thus, $R(x)$ has n+1 distinct roots, but its degree is $\leq n$. By the fundamental theorem of algebra, this is not possible unless $R(x) \equiv 0$. But this implies that:
\begin{center}
$P_n(x) = Q_n(x)$
\end{center}
Thus, there is only one polynomial of degree $\leq n$ that satisfies
\begin{center}
$P_n(x_i) = y_i, i = 0, ..., n$
\end{center}

\begin{problem}{2}
\end{problem}

\begin{enumerate}
	\item[a)]\begin{align*}
			P_1(x) &= f(x_0) + (x - x_0)f[x_0, x_1]\\
			&= 7 + (x - 0)(\frac{11-7}{2-0})\\
			&= 7 + 2x
		  \end{align*}
	\item[b)] \begin{align*}
			P_2(x) &= P_1(x) + (x - x_0)(x - x_1)f[x_0, x_1, x_2]\\
			&= 7 + 2x + (x - 0)(x - 2)(\frac{f[x_1, x_2] - f[x_0, x_1]}{x_2 - x_0})\\
			&= 7 + 2x + x(x-2)(\frac{17 - 2}{3-0})\\
			&= 7 + 2x + (x^2 - 2x)5\\
			&= 7 + 2x + 5x^2 - 10x = 5x^2 - 8x + 7
		  \end{align*}
	\item[c)] \begin{align*}
			P_3(x) &= P_2(x) + (x - x_0)(x - x_1)(x - x_2)f[x_0, x_1, x_2, x_3]\\
			&= 5x^2 - 8x + 7 + (x - 0)(x - 2)(x - 3)(\frac{f[x_1, x_2, x_3] - f[x_0, x_1, x_2]}{x_3 - x_0})\\
			&= 5x^2 - 8x + 7 + x(x-2)(x-3)(\frac{(\frac{f[x_2, x_3] - f[x_1, x_2]}{x_3 - x_1}) - 5}{4 - 0})\\
			&= 5x^2 - 8x + 7 + x(x-2)(x-3)(\frac{(\frac{\frac{63-28}{4-3} - 17}{4 - 2}) - 5}{4})\\
			&= 5x^2 - 8x + 7 + x(x-2)(x-3)(\frac{9 - 5}{4})\\
			&= 5x^2 - 8x + 7 + x(x-2)(x-3)\\
			&= 5x^2 - 8x + 7 + (x^2 - 2x)(x - 3)\\
			&= 5x^2 - 8x + 7 + x^3 - 3x^2 - 2x^2 + 6x = x^3 - 2x + 7
		  \end{align*}
\end{enumerate}

\begin{problem}{3}
\end{problem}

\noindent Since $n = 5$ and $f(x) = sin(x)$, $f^{n+1}(c) =  f^{6}(c) = -sin(c)$. Furthermore, we know $(n+1)! = 6! = 720$. Thus,
\begin{center}
$|\frac{f^6(c)}{6!}| \leq \frac{1}{720} \approx 0.001388889$
\end{center}
Now we just need to bound the term $|\prod_{i=0}^{5} (x - x_i)|$.\\
\\
Let $h = \frac{1.6875}{5}$ and $x_i = ih$. Then,
\begin{align*}
|\prod_{i=0}^{5} (x - x_i)| &= |(x)(x - 1(1.6875))(x - 2(1.6875))(x - 3(1.6875))(x - 4(1.6875))(x - 5(1.6875))|\\
&= |x^6 - 25.3125x^5 + 242.051x^4 - 1081.22x^3 + 2221.91x^2 - 1642.1x|
\end{align*}
Let's graph this function to try to find its upper bound on [0, 1.6875] now.
\newpage
\begin{tikzpicture}[scale=2]
  \begin{axis}[
    xmin = 0, xmax = 1.6875,
    grid = both,
    domain=0:1.6875,
    ymin = 0,
    ymax = 450,
    yticklabel style = {font = \tiny, xshift = .5ex},
    xticklabel style = {font = \tiny, yshift = 0.5ex},
    samples=101,
    smooth,
    no markers,
    ]
    \addplot {abs(x^6 - 25.3125*x^5 + 242.051*x^4 - 1081.22*x^3 + 2221.91*x^2 - 1642.1*x)};
    \node at (120, 350) {\tiny$y = \left|\prod_{i=0}^{5} (x - x_i)\right|$};
    \draw [dashed, thick] (0, 400) -- (168.75,400);
    \node at (70, 420) {\tiny$y = 400$};
  \end{axis}
\end{tikzpicture}

\noindent As evident in the plot, 
\begin{center}
$y = |x^6 - 25.3125x^5 + 242.051x^4 - 1081.22x^3 + 2221.91x^2 - 1642.1x| \leq 400$
\end{center}

\noindent Thus,
\begin{align*}
\left|f(x) - p(x)\right| &= \left|\frac{f^6(c)}{6!} \prod_{i=0}^{5} (x - x_i)\right|\\
&\leq \left|\left(\frac{1}{720}\right)400\right| = \frac{5}{9} \approx 0.555556
\end{align*}
is an upper bound for the error $|f(x) - p(x)|$ on $[0, 1.6875]$.

\begin{problem}{4}
\end{problem}

\noindent From (4.64) in the textbook,
\begin{align*}
&\frac{1}{6}M_1 + \frac{2}{3}M_2 + \frac{1}{6}M_3 = 3\\
&\frac{1}{6}M_2 + \frac{2}{3}M_3 + \frac{1}{6}M_4 = 0\\
&\frac{1}{6}M_3 + \frac{2}{3}M_4 + \frac{1}{6}M_5 = -2
\end{align*}
\newpage
\noindent We know from (4.65) that $M_1 = M_5 = 0$. Combining this with the above system of equations yields:
\begin{center}
$M_2 = \frac{129}{28}$, $M_3 = -\frac{3}{7}$, $M_4 = -\frac{81}{28}$
\end{center}
Thus, substituting into (4.63), we obtain:
\begin{center}
$s(x) = \begin{cases} 
      \frac{43x^3}{56} - \frac{129x^2}{56} - \frac{13x}{28} + 5 & 1 \leq x \leq 2 \\
      \frac{-47x^3}{56} + \frac{411x^2}{56} - \frac{79x}{4} + \frac{125}{7} & 2 \leq x \leq 3 \\
      \frac{-23x^3}{56} + \frac{195x^2}{56} - \frac{229x}{28} + \frac{44}{7} & 3 \leq x \leq 4\\
      \frac{27x^3}{56} - \frac{81x^2}{14} + \frac{1213x}{56} - \frac{1201}{56} & 4 \leq x \leq 5
   \end{cases}
$
\end{center}

\begin{problem}{5}
\end{problem}

\begin{enumerate}[\alph*)]
\item Each $a_{ij}$ is comes from one of the following coefficients in (4.64):
\begin{center}
$\frac{x_j - x_{j-1}}{6}$, $\frac{x_{j+1} - x_{j-1}}{3}$, $\frac{x_{j+1} - x_j}{6}$
\end{center}

We know $x_1 < x_2 < ... < x_n$. Thus, in the equations above, we have:
\begin{center}
$x_j > x_{j-1}$, $x_{j+1} > x_{j-1}$, $x_{j+1} > x_j$
\end{center}

Thus, each numerator term is positive for any value $2 \leq j \leq n-1$. Since all the denominators are positive, each $a_{ij}$ is positive.\\

Now we need to check that $a_{ii} = 2(a_{ii-1} + a_{ii+1})$ where $a_{10} = a_{nn+1} = 0$.\\

Let's examine the base case,
\begin{align*}
a_{11} &= 2\left(a_{10} + a_{12}\right)\\
&= 2\left(0 + \frac{x_3 - x_2}{6}\right)\\
&= \frac{x_3 - x_2}{3}
\end{align*} 

However, we know that,
\begin{align*}
a_{11} = \frac{x_3 - x_1}{3} \neq \frac{x_3 - x_2}{3}
\end{align*}

since $x_2 \neq x_1$. Thus, the matrix does not satisfy this condition for row $1$. It can be shown that row $n$ also does not satisfy the condition.\\
\\
Let's try checking the interior rows $(2,...,n-1)$ to see if they satisfy the condition.
\begin{align*}
a_{ii} &= 2\left(a_{ii-1} + a_{ii+1}\right)\\
&= 2\left(\frac{x_{i+1} - x_{i}}{6} + \frac{x_{i+2} - x_{i+1}}{6}\right)\\
&= \frac{x_{i+2} - x_{i}}{3}
\end{align*}
Since the indeces for $\vec{x}$ correspond to the indices for $M$ offset by 1 (because we have $M_1 = M_n = 0$), we have $j = i+1$ where $j$ is the subscript for $x_j$ in the formulas in (a) and $i$ is the index of the row in the matrix. Plugging in that relationship yields,
\begin{align*}
a_{ii} &= \frac{x_{i+2} - x_{i}}{3}\\
&= \frac{x_{j+1} - x_{j}}{3}
\end{align*}

This is precisely the formula desired for each $a_{ii}$. Thus, the condition  $a_{ii} = 2(a_{ii-1} + a_{ii+1})$ holds for rows $2, ..., n-1$.


\item Suppose $A$ is invertible. Then we can write:
\begin{align*}
&A\vec{x} = \vec{b}\\
\implies &A^{-1}(A\vec{x}) = A^{-1}(\vec{b})\\
\implies &\vec{x} = A^{-1}\vec{b}
\end{align*}

Now let's show that $A^{-1}$ is unique. Suppose there are two matrices that are inverses of $A$ called $B$ and $C$. Then,
\begin{align*}
&AB = BA = AC = CA = I\\
\implies &B = B(I) = B(AC) = (BA)C = (I)C = C
\end{align*}

Thus, $B = C$ and the inverse of $A$ is unique.\\
\\
Since matrix multiplication is a deterministic algorithm and both $A^{-1}$ and $\vec{b}$ are unique, then there can only be one solution $\vec{x} = A^{-1}\vec{b}$.

\item For rows $2, ..., n-1$, we have
\begin{align*}
&a_{ii-1}x_{i-1} + a_{ii}x_i + a_{ii+1}x_{i+1} = 0\\
\implies &a_{ii-1}x_{i-1} + 2x_i(a_{ii-1} + a_{ii+1}) + a_{ii+1}x_{i+1} = 0
\end{align*}

\item We have,
\begin{align*}
a_{11}x_1 + a_{12}x_2 = 0
\end{align*}

\end{enumerate}
\end{document}